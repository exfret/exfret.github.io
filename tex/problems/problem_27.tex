
\section{Linear progressions without the digit 7}

Author: Victor Wang --- ID: \verb`2346197958958131` --- Date: 11/28/2018

\subsection{Problem statement}

Consider a real number $r$ and the sequence of multiples of $r$ (i.e.- $(r,2r,3r,\ldots)$). Show that this sequence eventually contains a number that has the digit $7$ in its decimal representation.

Is there a natural number $n$ so that for every $r$ the set $\{r,2r,\ldots,nr\}$ must contain a number with the digit 7? If so, what is the smallest such $n$?

\subsection{Clarifications and comments}

Some numbers have two decimal representations. In this case, choose the larger one (or choose in another consistent way to your liking; it shouldn't matter too much).

\subsection{Progress}

It is known that the smallest such $n$ does exist and in fact equals 42 by some casework.

\subsection{Related problems}

We can of course wonder about other bases and other digits besides 7. I wonder whether there is some sort of general formula or algorithm for solving the second question about the minimum $n$.

