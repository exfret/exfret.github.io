
\section{A variation of Naughts and Crosses (Tic Tac Toe)}

Author: Ben Crossley --- ID: \verb`3131704127074173` --- Date: 3/16/2021

\subsection{Problem statement}

Two players in a game of tic tac toe (also known as naughts and crosses) are each given six pieces of different sizes. On a player's turn, they can place a piece in any empty square, or place one of their larger pieces on top of an opponent's piece. Someone wins if they keep three pieces in a row for a full round of play. What is the optimal strategy?

\includegraphics[scale=0.3]{Media/tic_tac_toe_variation_1.jpeg}

\subsection{Clarifications and comments}

\begin{enumerate}
  \item Players may only place pieces on a square that is empty or has an enemy piece on top. Players may not ``cover up their own piece".
  \item If a player can no longer place any pieces (either by running out or having pieces that are too small to place anywhere), then they must pass. A player can only pass in this situation. Two passes in a row result in a tie (unless one player has a three in a row).
  \item I don't know if this is possible, but if two three in a rows occur, then the first one to have appeared wins.
\end{enumerate}

\subsection{Progress}

A variation of the game later posted by Max Suica was solved via a computer brute force minimax algorithm, but I can't find Max's post again and forget the result.

\subsection{Related problems}

We can in general consider any set of pieces $S$ relation $R\subset S\times S$ where $R(x,y)$ if piece $x$ can be placed on $y$.

