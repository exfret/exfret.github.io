
\section{Number of vector orderings in a plane}

Author: Ben Millwood --- ID: \verb`3399730876938162` --- Date: 2/23/2022

\subsection{Problem statement}

Let $X\subset\mathbb{R}^n$ be the set of all vectors in $\mathbb{R}^n$ with distinct entries. Consider the function $f:X\to Sym_n$ defined by letting $f(v)$ be the unique permutation $\sigma$ in the symmetric group on $n$ letters such that $v_{\sigma(1)}<v_{\sigma(2)}<\cdots< v_{\sigma(n)}$. What is the maximum number of values that the intersection of a two dimensional subspace of $\mathbb{R}^n$ with $X$ can take under the function $f$?

\subsection{Clarifications and comments}

The function $f$ simply describes the ordering of the entries of a vector $v$. Vectors with the same orderings of their entries are assigned the same permutation, whereas those with different orderings of their entries are assigned different entries. For example, $(1,4,2)$ and $(-1,3,2.5)$ have the same orderings, but $(1,4,2)$ and $(6,4,2)$ do not, since the first entry is less than the second in the former vector and less than the second entry in the latter vector.

So, since a two dimensional subspace is just a space generated by two linearly independent vectors, this is essentially asking for the number of different orderings of the entries of a vector one can get from the linear combinations of two fixed vectors $u,v$.

\subsection{Progress}

The answer is $n(n-1)$.

Credit to Andrew Kepert for the following.

For a better upper bound, we can rotate around a circle in the two-dimensional subspace and count the number of times two entries ``switch their position in the ordering". That is, points $p$ where $v_i>v_j$ immediately beforehand, and where $v_i<v_j$ immediately after (for some $i\neq j$). It's easy to see that such points occur whenever $(e_i-e_j)\bullet v>0$ immediately before $p$ as we rotate around the circle, and $(e_i-e_j)\bullet v<0$ immediately after, which by the intermediate value theorem implies that $(e_i-e_j)\bullet p=0$. Since $g(v)=(e_i-e_j)\bullet v$ is a linear function, it either intersects the circle at no points, two points, or all points. If it intersects at all points, then the values of $v_i$ and $v_j$ never switch as we go around the circle. Thus, the maximum number of such points for each pair of $i,j$ is two, which gives an upper bound of $n(n-1)$.

Credit to me for the following. (Andrew Kepert also produced a lower bound proof, but I think this is more sleek).

Now for a lower bound. Consider a two-dimensional subspace $V$ generated by $u,v$ that have unique entries. For any such $u$ (or $v$ for that matter), the value of $(e_i-e_j)\bullet u\neq 0$ for any $i,j$, so the subspace $W_{i,j}$ of all points $p$ such that $(e_i-e_j)\bullet p=0$ does not contain $V$. So the intersection of $V$ and $W_{i,j}$ is not dimension two. It is also not dimension zero since $dim(V)=2,dim(W_{i,j})=n-1$.

All that remains to be shown is that the intersection of any $W_{i,j}$ with $W_{i',j'}$ is never contained in $V$, since then each $W_{i,j}$ will separate $V$ into two new regions, thus achieving the bound. But for any such $i,j$, the set of such two-dimensional subspaces (with the measure induced by randomly choosing two vectors in the unit hypersphere to generate said subspace) is zero, and the set of subspaces that can't be generated by $u,v$ with distinct entries is also zero, so there must must be a two-dimensional subspace that meat our criteria (in fact, almost all subspaces do).

\subsection{Related problems}

We can also consider $k$-dimensional subspaces for arbitrary $k$. The proofs above break down in some places, but are probably still recoverable.

