
\section{Multiples with digits only 0's and 1's}

Author: Rose Horner --- ID: \verb`2317179215193339` --- Date: 10/19/2018

\subsection{Problem statement}

\begin{enumerate}
    \item (Easy) Any integer in base 10 ending in 1 has a multiple containing only 1's.
    \item (Hard) Any integer has a multiple containing only zeroes and ones.
    \item (Harder) Find the smallest such multiple in terms of the given integer $n$.
\end{enumerate}

\subsection{Clarifications and comments}

None.

\subsection{Progress}

The hard problem can be solved in a way that implies the easy problem (I think there's a more direct method, but this is the best I know of as of writing this). Credit to Kevin An for the proof and to Michael Yu for observing its extension to the easy version.

Given a number $n$ imply consider the set $\{1,\ldots,1\cdots 1\}$ where the last number is just $1$ written $n+1$ times. By pigeonhole principle, there are two numbers with the same residue mod $n$, so simply take there difference, which will be a sequence of 1's then a sequence of 0's.

To extend to the easy variant, simply note that ending in 1 implies that the number can't be divisible by 2 or 5, and thus we can appeal to the fundamental theorem of arithmetic to remove all factors of $10=2\times 5$ from our sequence of 1's and 0's found in the last paragraph while still having it be divisible by $n$ (thus removing all the zeroes and keeping only ones).

For the harder answer, a formula involving the multiplicative order can be found, but a more explicit one has not been found (and probably won't considering the use of the multiplicative order function).

\subsection{Related problems}

We can also consider restrictions to other sets of digits.

