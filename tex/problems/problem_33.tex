
\section{Digit natural density operator}

Author: Max Weinrich --- ID: \verb`2306329172945010` --- Date: 10/3/2018

\subsection{Problem statement}

For any number $x\in[0,1]$ let $z(x)$ be the natural density of the number of $0's$ in the decimal expansion of $x$ (if it exists).

\begin{enumerate}
\item What is the measure of the level set $z^{-1}(p)$ for various values of $p\in[0,1]$?
\item Can you name a point $x$ such that the sequence $x,z(x),z(z(x)),\ldots$ exists and never repeats a value?
\end{enumerate}

\subsection{Clarifications and comments}

The natural density of a set $S\subset\mathbb{N}$ is the value of the limit $\lim_{n\to\infty}\frac{\lvert S\cap[1,n]\rvert}{n}$ (i.e.- we take the limit of the fraction of numbers $\leq n$ that are in $S$ as $n$ tends to infinity). In the above problem $S$ is the subset of numbers $i\in\mathbb{N}$ such that the $i$th digit of $x$ is a zero.

Note that some numbers have multiple decimal expansions, so we can just choose the largest one (or smallest if that's to your fancy).

\subsection{Progress}

The first question is fairly easy: almost all numbers are normal, which implies that almost all numbers have a 1/10 fraction of their digits as 0's. Thus, the answer is 1 for $p=1/10$ and 0 otherwise.

The second question is open.

\subsection{Related problems}

We can also study other questions/properties of this function, or study the properties of the function with respect to other bases.

